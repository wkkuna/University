\documentclass{article}
\usepackage[utf8]{inputenc}
\usepackage{polski}
\usepackage{mathtools}
\usepackage[a4paper, total={6in, 8in}]{geometry}
\usepackage{textcomp}
\usepackage{amssymb}
\usepackage{listings}

\font\myfont=cmr12 at 30pt

\title{\myfont Zadanie 4}
\author{Wiktoria Kuna}
\date{316418}

\begin{document}
\setlength{\parindent}{0.5ex}
\setlength{\parskip}{1.5ex}
\maketitle

\section*{Polecenie}

Rozkład t-studenta z $k$ stopniami swobody ma gęstość:

\begin{gather*}
    f(x) = \frac{\Gamma{(\frac{k+1}{2} )}}{\sqrt{k \pi} \Gamma{(\frac{k}{2})}}\Big(1 + \frac{x^2}{k}\Big)^{-\frac{k+1}{2}}, \, x \in \mathbb{R}
\end{gather*}
Dla ustalonego $t \in \mathbb{R}$ obliczyć wartość całki:
\begin{gather*}
    G(t) = \int_{-\infty}^t f(x) \, dx
\end{gather*}




\section*{Parzystość gęstości}

Zauważmy, że gęstość rozkładu $t$-studenta jest funkcją parzystą. 

Dowód:

\begin{gather*}
    f(x) = \frac{\Gamma{(\frac{k+1}{2} )}}{\sqrt{k \pi} \Gamma{(\frac{k}{2})}}\Big(1 + \frac{x^2}{k}\Big)^{-\frac{k+1}{2}} = \frac{\Gamma{(\frac{k+1}{2} )}}{\sqrt{k \pi} \Gamma{(\frac{k}{2})}}\Big(1 + \frac{(-x)^2}{k}\Big)^{-\frac{k+1}{2}} = f(-x)
\end{gather*}

\section*{Lemat: G(t) = 1 - G(-t)}

Dowód:
\begin{gather*}
    G(t) = \int_{-\infty}^t f(x) \, dx \stackrel{(1)}{=} 
    \int_{-\infty}^t f(-x) \, dx = 
    \int_{-t}^{\infty} f(x) \, dx =\\
    \int_\mathbb{R} f(x) -  \int_{-\infty}^{-t} f(x) \, dx = 1 - G(-t)
\end{gather*}


$(1)$ Korzystamy z parzystości gęstości rozkładu t-studenta.

\section*{Przekształcenia}
Aby móc skorzystać ze złożonego wzoru trapezów i metody Romberga, całka którą liczymy musi być właściwa. Musimy zatem przeprowadzić odpowiednie przekształcenia. 

\begin{gather*}
    G(t) \stackrel{Lemat}{=} 1 - G(-t) = 1 -\int_{-\infty}^{-t} f(x) \, dx = \\
    1 -\int_{-\infty}^{-t} \frac{\Gamma{(\frac{k+1}{2} )}}{\sqrt{k \pi} \Gamma{(\frac{k}{2})}}\Big(1 + \frac{x^2}{k}\Big)^{-\frac{k+1}{2}} \, dx =\\
    1 - \int_{-\infty}^{-t} \frac{\Gamma{(\frac{1}{2} (k+1) )}}{\sqrt{k \pi} \Gamma{(\frac{1}{2} k)}}\Big(1 + \frac{x^2}{k}\Big)^{-\frac{k+1}{2}} \, dx\\
\end{gather*}

Wiemy, że $\Gamma{(\frac{1}{2})} = \sqrt{\pi}$. 

\begin{gather}
    1 - \int_{-\infty}^{-t} \frac{\Gamma{(\frac{1}{2} (k+1) )}}{\sqrt{k} \Gamma{(\frac{1}{2})} \Gamma{(\frac{1}{2} k)}}\Big(1 + \frac{x^2}{k}\Big)^{-\frac{k+1}{2}} \, dx
\end{gather}

Na ćwiczeniach udowadnialiśmy następującą zależność: $\Gamma{(p)} \Gamma{(q)} = \Gamma{(p+q)} B{(p,q)}$.

Niech $p = \frac{1}{2}$ oraz $q = \frac{1}{2}k$. Z powyższej zależności mamy:

\begin{gather*}
\Gamma{\Big(\frac{1}{2}\Big)} \Gamma{\Big(\frac{1}{2}k\Big)} = \Gamma{\Big(\frac{1}{2} (k+1)\Big)} B{\Big(\frac{1}{2},\frac{1}{2}k\Big)}\\
\frac{\Gamma{\Big(\frac{1}{2} (k+1)\Big)}}{\Gamma{\Big(\frac{1}{2}\Big)} \Gamma{\Big(\frac{1}{2}k\Big)} } = \frac{1}{B{\Big(\frac{1}{2},\frac{1}{2}k\Big)}}
\end{gather*}

Teraz możemy podstawić powyższą zależność do $(1)$: 

\begin{gather*}
    1 - \int_{-\infty}^{-t} \frac{1}{B{\Big(\frac{1}{2},\frac{1}{2}k\Big) \sqrt{k}}}\Big(1 + \frac{x^2}{k}\Big)^{-\frac{k+1}{2}} \, dx =\\
    1 - \frac{1}{B{\Big(\frac{1}{2},\frac{1}{2}k\Big) \sqrt{k}}} \int_{-\infty}^{-t} \Big(1 + \frac{x^2}{k}\Big)^{-\frac{k+1}{2}} \, dx 
\end{gather*}

\pagebreak

Dokonajmy teraz podstawienia za $x$. Od teraz rozpatrujemy wyłącznie $t \geq 0$.

$$
    \begin{vmatrix}
    y = \frac{k}{x^2 + k}\\ \\
    |x| = \sqrt{\frac{k(1-y)}{y}}  \implies x = -\sqrt{\frac{k(1-y)}{y}} \, (x \leq 0)\\ \\
    dx = - \frac{1}{2} (\frac{k(1-y)}{y})^{-\frac{1}{2}} \cdot \frac{-ky - k(1-y)}{y^2}  dy = \\ \\
    -\frac{1}{2} (\frac{k(1-y)}{y})^{-\frac{1}{2}} \cdot \frac{-k}{y^2}  dy
    \end{vmatrix} 
$$

Wraz z podstawieniem musimy oczywiście zmienić granice całkowania. Skoro $x$ znajdował się na przedziale $(-\infty, -t]$, a $y = \frac{k}{x^2 + k}$ to $y$ będziemy całkować na przedziale $[0, \frac{k}{t^2 + k}]$.

\begin{gather*}
        1 - \frac{1}{B{\Big(\frac{1}{2},\frac{1}{2}k\Big) \sqrt{k}}} \int_{0}^{\frac{k}{t^2 + k}} \Big(1 + \frac{\frac{k(1-y)}{y}}{k}\Big)^{-\frac{k+1}{2}} \frac{-1}{2} \Big(\frac{k(1-y)}{y}\Big)^{-\frac{1}{2}} \cdot \frac{-k}{y^2}  \, dy =\\ 
         1 - \frac{1}{B{\Big(\frac{1}{2},\frac{1}{2}k\Big) \sqrt{k}}} \frac{1}{2}  
         \int_{0}^{\frac{k}{t^2 + k}} 
         \Big(1 + \frac{(1-y)}{y}\Big)^{-\frac{k+1}{2}} 
         \Big(\frac{k(1-y)}{y}\Big)^{-\frac{1}{2}} \cdot \frac{k}{y^2}  \, dy =\\ 
        1 - \frac{1}{B{\Big(\frac{1}{2},\frac{1}{2}k\Big) \sqrt{k}}} \frac{1}{2}  
        \int_{0}^{\frac{k}{t^2 + k}} 
        \Big(\frac{1}{y}\Big)^{-\frac{k+1}{2}} 
        \Big(\frac{k(1-y)}{y}\Big)^{-\frac{1}{2}} \cdot 
        \frac{k}{y^2}  \, dy =\\ 
        1 - \frac{1}{B{\Big(\frac{1}{2},\frac{1}{2}k\Big) }} \frac{1}{2}  
        \int_{0}^{\frac{k}{t^2 + k}} 
        \Big(y\Big)^{\frac{k}{2} - 1} \cdot y ^{\frac{1}{2}} \cdot
        \Big(\frac{y}{1-y}\Big)^{\frac{1}{2}} \cdot 
        \frac{1}{y}  \, dy =\\ 
        1 - \frac{1}{B{\Big(\frac{1}{2},\frac{1}{2}k\Big) }} \frac{1}{2}  
        \int_{0}^{\frac{k}{t^2 + k}} 
        \Big(y\Big)^{\frac{k}{2} - 1} \cdot
        \Big(\frac{1}{1-y}\Big)^{\frac{1}{2}} \, dy =\\ 
        1 - \frac{1}{B{\Big(\frac{1}{2},\frac{1}{2}k\Big) }} \frac{1}{2}  
        \int_{0}^{\frac{k}{t^2 + k}} 
        \Big(y\Big)^{\frac{k}{2} - 1} \cdot
        \Big(1-y\Big)^{\frac{1}{2} - 1} \, dy =\\ 
        1 - \frac{1}{2} \frac{B\Big(\frac{k}{t^2 + k}; \frac{k}{2}, \frac{1}{2}\Big)}{B{\Big(\frac{1}{2}k, \frac{1}{2}\Big) }} 
\end{gather*}

W ten sposób do policzenia mamy funkcję Beta i niepełną funkcję Beta, czyli dwie całki właściwe.

\pagebreak

Wyprowadziliśmy postać dla $t \geq 0$, jednak wcześniej pokazaliśmy, że rozkład $t$-studenta jest symetryczny, stąd mamy równocześnie wartości dla $t \leq 0$. 

Mamy zatem:
\begin{gather*}
G(t) = 
\begin{cases}
    1 - \frac{1}{2} \frac{B\Big(\frac{k}{t^2 + k}; \frac{k}{2}, \frac{1}{2}\Big)}{B{\Big(\frac{1}{2}k, \frac{1}{2}\Big) }}  & t \geq 0\\
    \frac{1}{2} \frac{B\Big(\frac{k}{t^2 + k}; \frac{k}{2}, \frac{1}{2}\Big)}{B{\Big(\frac{1}{2}k, \frac{1}{2}\Big) }}  & t \leq 0
\end{cases}
\end{gather*}

\section*{Metoda Romberga}

Metoda Romberga polega na rekurencyjnym przybliżaniu wartości funkcji zaczynając od aporksymacji z złożonej metody trapezów.

Wzory realizujące metodę:
\begin{gather*}
h_n = {b-a \over 2^n}\\
R(0,0) = h_1 \cdot (f(b)+f(a))\\
R(n,0) = R(n-1,0) +  h_n \cdot \sum_{k=1}^{2^{(n-1)}} f(a+(2\cdot k-1) \cdot h_n)\\
R(n,m) = {1 \over 4^m -1} \cdot (4^m \cdot R(n,m-1) + R(n-1,m-1))\\
\end{gather*}

\pagebreak

\section*{Program}

Do obliczenia całki wykorzystamy złożony wzór trapezów i metodę Romberga.
Do otrzymania wartości rozkładu $t$-studenta potrzebna nam watrość funkcji Beta i niekompletnej funkcji Beta. Do pierwszej użyjemy funkcji bibliotecznej, natomiast drugą policzymy metodą Romberga.

\begin{lstlisting}[language=Python]
import scipy.integrate as inte
import scipy.special as sc

def fun(a, b):
    def iner(y):
        if y == 0:
            return 0.
        if y == 1:
            return 0.
        return y**(a - 1) * (1-y)**(b-1)
    return iner

def Tstudent_beta(k):
    return sc.beta(k/2., .5)

def Tstudent_beta_incomplete(k, t):
    f = fun(k/2., 1/2.)
    return inte.romberg(f, 0., k/(t**2+k), divmax=50, tol=10**(-8), show=True)

def Tstudent_CDF(k,t):
    romb = 0.5 * Tstudent_beta_incomplete(k,t)/Tstudent_beta(k)
    if t < 0:
        return romb
    else:
        return 1 - romb
\end{lstlisting}

Wartość divmax określa maksymalną liczbę kroków, na którą pozwalamy metodzie Romberga. Osiągnięcie jednak dokładności do $8$ miejsca po przecinku może zająć dużo czasu, gdyż wymaga wielu iteracji.

\section*{Dokładność}

Python zapewnia, że biblioteczna funkcja Beta wyliczy się z dokładnością do $10^{-14}$. Jeśli damy wystarczająco czasu na wykonanie się programu, procedura korzystająca z metody Romberga da nam dokładność $10^{-8}$ (w wypadku niecierpliwym $10^{-5}$). Stąd jesteśmy w stanie wyliczyć wartość $G(t)$ z dokładnością do 8 miejsc po przecinku.
\end{document}
